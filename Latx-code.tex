\documentclass{article}
\usepackage{amsmath, amsthm, amsfonts, amssymb, geometry, natbib}

% Set margin to 2.5cm on all sides
\geometry{margin=2.5cm}

\title{Exercise Set}
\author{Walid Alwahabi}
\date{April 24, 2024}

\begin{document}

\maketitle

\tableofcontents

\section{Introduction}
This exercise will focus on the different ways you use Latex 

\subsection{Formatting Text}
In this section, we demonstrate formatting text using bold and italic styles:
\begin{itemize}
    \item \textbf{Bold text}
    \item \textit{Italic text}
\end{itemize}

\section{Mathematical Notation}
Here is an example equation using the equation environment:
\begin{equation*}
    E = mc^2
\end{equation*}

\section{BibTeX}
    \citep{mason_2006_elearning}
    \bibliographystyle{agsm}
    \bibliography{references}
\section{Table Example}
\begin{table}[h]
    \centering
    \begin{tabular}{|c|c|}
        \hline
        Row 1, Col 1 & Row 1, Col 2 \\
        \hline
        Row 2, Col 1 & Row 2, Col 2 \\
        \hline
    \end{tabular}
    \caption{Example Table}
\end{table}

\section{Font families}
\text{Serif font family:} This text is in serif font.
\textsf{Sans-serif font family:} This text is in sans-serif font.

\texttt{Mono-spaced font family:} This text is in mono-spaced font.

\section{Rulers}
\noindent\rule{\textwidth}{0.5pt}

\section{Spaces}
\subsection{Horizontal Spaces}
This is some text.\hspace{2cm}This is text with extra horizontal space.

\subsection{Vertical Spaces}
This is some text.

\vspace{1cm}

This is text with extra vertical space.

\section{Theorems and Proofs}
\subsection{Pythagorean Theorem}
\newtheorem{theorem}{Pythagorean Theorem}
\begin{theorem}
Given a right-angle triangle with hypotenuse of length $c$ and side lengths $a$ and $b$, we have:
    \begin{equation}
        c^2 = a^2 + b^2
    \end{equation}
\end{theorem}

\begin{proof}
Now we start with four copies of the same triangle. Three of these have been rotated 90°, 180°, and 270°, respectively. Each has area $\frac{ab}{2}$. Let's put them together without additional rotations so that they form a square with side $(a + b)$ and a square hole with side $c$.

\begin{equation}
    (a + b)^2 = 4 \frac{ab}{2} + c^2
\end{equation}
\begin{equation}
    a^2 + 2ab + b^2 = 2ab + c^2
\end{equation}
\begin{equation}
    a^2 + b^2 = + c^2
\end{equation}

\end{proof}

\end{document}
